
\documentclass[12pt]{article}
\usepackage[utf8]{inputenc}
\usepackage[spanish]{babel}
\usepackage{amsmath} 
\usepackage{amssymb} % utlizar mathbb


\newenvironment{micaja}
{
    \begin{center}
    \begin{tabular}{|p{0.9\textwidth}|}
    \hline\\
    }   
    {   
    \\\\\hline
    \end{tabular} 
    \end{center}
    }



\title{Ejercicios tema 4. \\
Grupos cociente. Teoremas de isomorfismo.Productos.}
\author{Blanca Cano Camarero}

\begin{document}
\begin{titlepage}
\maketitle
\tableofcontents
\end{titlepage}

\section{Ejercicio 10}

\begin{micaja}
    Sean $H$,$K$ dos subgrupos finitos del grupo $G$, uno de ello, $H$ normal. Demostrar que 
    $$|H||K| = |HK||H \cap K|$$
\end{micaja}

Por el tercer teorema de isomorfía sabemos que 
$$\frac{K}{H \cap K} \cong \frac{KH}{H}$$
de lo que deducimos que el respectivo número de clases laterales el mismo, es decir que 
$$[K: H \cap K] = \left|\frac{K}{H \cap K} \right| =  \left| \frac{KH}{H} \right| = [KH:H].$$

Además por el teorema de Lagrange sabemos que: 

$|K| = [K: H \cap K] |H \cap K|$ y que $|HK| = [KH:H]|H|$

Por tanto, combinando estas igualdades, podemos desmejar la igualdad que se nos pedía demostrar. 

$$\frac{|K|}{|H \cap K|} = \frac{|HK|}{|H|}$$

\newpage

\section{Ejercicio 11}

\begin{micaja}
    sea $N \trianglelefteq G.$ Probar que $G/N \cong G$ si y solo si $N=\{1\}.$\\
     Probar que $G/N \cong \{1\}$ si y solo si $N=G$
\end{micaja}
\subsection*{Condiciones suficientes}
Gracias al teorema de lagrage sabemos que $|G| = |G/N||N|$
Como  por hipótesis  $G/N \cong G$ entonces $|G/N| = |G|$. 
De estas dos igualdades anteroriores deducimos que: $|N|=1$ y 
como la unidad pertenece por ser subgrupo no queda más remedio que $N=\{1\}.$

Si por el contrario, nuestra hipótesis fuera $G/N \cong \{1\}$ entonces $|N| = |G|$, 
y el único subgrupo  de $G$ que tiene la misma cardinalidad es él mismo, es decir $N=G.$

\subsection*{Condiciones necesarias}



\newpage

\section{Ejercicio 23. }
\begin{micaja}
Sean G, H y K grupos. 
Demostrar que: 
$$H \times K \cong K \times H.$$  
$$G \times (H \times K) \cong (G \times H) \times K.$$
\end{micaja}

\subsubsection*{Primer isomorfismo}

Definamos $f: H \times K  \rightarrow K \times H, $ de la forma $f(h,k) = (k,h).$
Veamos que esto es un isomorfismo: 

\begin{itemize}
    \item Epimorfismo: Sean $(k_1,h_1), (k_2,h_2) \in K \times H$ 
    tales que $f^{-1}(k_1,h_1)=f^{-1}(k_2,h_2)$ entonces por cómo se ha definido la aplicación
    se tiene que $(k_1,h_1)=(k_2,h_2).$
    \item Monomorfismo: $f^*(1,1) = \{(1,1)\}$. 

\end{itemize}

\subsubsection*{Segundo isomorfismo}.

No consigo entender qué hay que probar ¿no es evidente por cómo se define
el producto directo que $G \times H \times K \cong G  \times (H \times K) \cong (G \times H) \times K.$?


\newpage

\section{Ejercicio 26}
\begin{micaja}

    No todo subgrupo de un grupo directo es producto directo de subgrupos.
\end{micaja}   

Ejemplo dado el grupo $(\mathbb Z_2, \times),$ para $\mathbb Z_2 \times \mathbb Z_2$ se tiene que $<(1,1)>= \{(0,0),(1,1)\}$
no es ningún subgrupo generado por subgrupos de $\mathbb Z_2,$ ya que 
estos son: 
\begin{center}
    \begin{tabular}{c c l}
$\{0\} \times \{0\}$ & = &$ \{(0,0)\}$ \\
$\{0\} \times \mathbb Z_2$ & = &$ \{(0,0),(0,1)\}$ \\
$\mathbb{Z}_2 \times \{0\}$ & = &$ \{(0,0),(1,0)\}$ \\
$\mathbb{Z}_2 \times \{0\}$ & = &$ \{(0,0),(1,0),(0,1)(1,1)\}$    
    \end{tabular}
\end{center}

El recíproco sí es cierto.
\end{document}
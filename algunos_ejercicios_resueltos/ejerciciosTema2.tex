\documentclass[12pt]{article}
\usepackage[utf8]{inputenc}
\usepackage[spanish]{babel}
\usepackage{amsmath} 
\usepackage[dvipsnames]{xcolor}

\newenvironment{micaja}
{
    \begin{center}
    \begin{tabular}{|p{0.9\textwidth}|}
    \hline\\
    }   
    {   
    \\\\\hline
    \end{tabular} 
    \end{center}
    }



\title{Ejercicios tema 2}
\author{Blanca Cano Camarero}

\begin{document}
\begin{titlepage}
\maketitle
\tableofcontents
\end{titlepage}

\section[Ejercicio 9]{Ejercicio 9}
\begin{micaja}
Sea $G$ un grupo y sean $a,b \in G$ tales que $ba = ab^k , a^n = 1 = b^m$
con $n,m>0$
\end{micaja}

Demuestre 
\begin{enumerate}
 
    % APARTADO 1 __________________________
    \item \textbf{Para todo $i=0,...,m-1$ se verifica $b^i a = a b^{ik}$}
    
    Demostraremos por inducción que se cumple para todo $i=0,...,m-1$.

    \begin{itemize}
        \item Caso base $i = 0$ evidente
        \item Supongamos que para $i \in \{0...m-2\}$ se cumple y veamos que para i+1 también 
        
        $b^{i+1} a = b b^i a$ usando hipótesis de inducción y después hipótesis iniciales llegamos a  
        $b (b^i a) = (b a) b^{ik}  = a  b^k b^{ik} = a ^{(i+1)k}$  
        Como se quería probar. 
    \end{itemize}  

    % APARATADO 2  __________________
    \item \textbf{Para todo $j  = 0,...,n-1$ se vereifica $b a^j = a^j b^{k^j}$}
    
    Lo demostraremos por  inducción que se cumple para todo $j=0,...,n-1$: 

    \begin{itemize}
        \item Caso base $j = 0$ evidente.
        \item Supongamos que para $j \in \{0...n-2\}$ se cumple y veamos que para j+1 también 
        $ b a^{j+1} = (b  a ^j ) a$ . 
        Por la hipótesis de inducción $(b  a ^j ) a = a^j b^{k ^j} a$  \\
        Usando ahora el partado 1 de este mismo ejercicio. \\
        $a^j (b^{k ^j} a) = a^j a b^{k^j k } =  a^{j+1}b^{k^{j+1}}$  \\

        Como queríamos ver 
        
    \end{itemize}  

    % APARTADO 3______

    \item \textbf{Para todo i=0,..,m-1 y para todo j = 0,..n-1 se verifica $b^ia^j = a^j b^{i k ^j}$.}
    
    Sea un j cualquiera en el rango mencionado. 
    Lo demostraremos por inducción sobre i. 

    \begin{itemize}
        \item Caso base i=0, evidente para cualquier j. 
        \item Hipótesis de inducción, supongamos cierto para  $1\leq i < m-1$ y veamo que se cumple para el caso $i+1$. 
        
        $b^{i+1} a^{j} =  b (b^i a^j) = (b a^j) b^{i k^j} = a^j b^{k^j} b^{i k^j} = a^j b^{(i+1) k^j}$
        Donde para la primera igualadas se ha usado la prompiedad asociativa, para la segunda la hipótesisi de inducción, para la tercera el apartado 2 
        de este mismo ejercicio y finalmente otra vez la propiedad asociativa. 

        Como el j era arbitrario hemos probado lo que queríamos. 
    \end{itemize}


    % APARTADO 4 _______________
    \item \textbf{Demostrar que todo elemento de $<a,b>$ puede escribirse como $a^rb^a$ con $0 \leq s <m$ $0 \leq r <n$.}
    
    Esto es consecuencia directa del apartado anterior y la definición de grupo generado, ya que todo elemento tendrá la forma $a^{z_0} b^{z_1}a*{z_2}...$ con 
    los exponenetes enteros. Por tanto utilizando el apartado 3 podremos quedarno con que  $a^{z_0} b^{z_1}a*{z_2}... = a^x b^y$ con x,y enteros. 

    Y como sea cual sera x e y podemos expresarlos como $ x = pn r, y = qm +s $
    Usando las hipótesisi iniciales probamos con ello lo que queríamos. 


\end{enumerate}
\newpage

\section[Ejercicio 10]{Ejercicio 10 }
\begin{micaja}
Demostrar que un subconjunto no vacío $X \subseteq G$ de un grupo $G$ es un subgrupo si y solo si $X = <X>$.
\end{micaja}


\textbf{Condición suficiente}
Si $y \in <X>$ entonces tendrá la forma de producto de elementos de $X$ (definición de grupo generado),
 como $X$ es subgrupo entonces será cerrado para el producto y tenemo por tanto que $X = <X>$.

 \textbf{Condición necesaria}
Para cualesquiera $a,b \in X$ se tiene que $ab^{-1} \in <X> = X$ (por hipótesis) entonces acabamos de probar
que $X$ es subgrupo. 
\newpage

\section[Ejercicio 13]{Ejercicio 13}

    \begin{micaja}
    1. Demostrar que si $H \leq G$ es un subgrupo, entonces $[G:H] = |G|$ si y solo si, $H=\{1\}$, mientras que 
    $[G:H]=1$ sii $H=G$.
    \end{micaja}

    Todo esto es consecuencia inmediata del teorema de Lagrange.
    
    
\underline{Para $[G:H] = |G|$ si y solo si, $H=\{1\}$}

    \begin{itemize}
        \item 
   
    \textbf{Condición necesaria}. Por ser $H$ subgrupo distinto del vacío $1\in H$. El teorema de lagrange nos dice 
    que $[G:H] |H| = |G|$, entonces tenemos que $|H|=1$ y esto implica que $H=\{1\}$.

    \item \textbf{Condición suficiente}. Si $H=\{1\}$ entonces $|H|=1$ y por el teorema de lagrange $[G:H] = |G|$.
    \end{itemize}

    \underline{Para $[G:H]=1$ sii $H=G$.}
    \begin{itemize}
    \item \textbf{Condición necesaria}. Como  $H \leq G$ y por el teorema de lagrange $|H| = |G|$ entonces   $H=G$.
    \item\textbf{Condición suficiente}. Si $H=G$ entonces $|H| = |G|$ y por el teorema de lagrangre no nos queda más que $[G:H]=1$, 
    como queriamos probar. 
\end{itemize}   
    \begin{micaja}
    2. Demostrar que si se tienen los subgrupos $G_2 \leq G_1 \leq G$, entonces $|G:G_2| = [G:G_1][G1:G2]$
    \end{micaja}
Por la transitividad de ser subgrupo y el teorema de lagrange, llegamos a las siguientes igualdades:
    \begin{align}
        |G| = [G:G_1] |G_1|
    \end{align}
    \begin{align}
        |G| = [G:G_2] |G_2|
    \end{align}
    \begin{align}
        |G_1| = [G_1:G_2] |G_2|
    \end{align}

    Sustituimos en la primera igualdad la el valor de $|G|$ que nos da la segunda y para el mienbro de la derechas Sustituimos
    el valor de $|G_1|$ por el que nos da la tercera igualdad obteniendo la siguiente ecuación. 

    $[G:G_2] |G_2| = [G:G_1] [G_1:G_2] |G_2|$, por tanto hemos probado lo que buscábamos

    $$[G:G_2] |G_2| = [G:G_1] [G_1:G_2] |G_2|$$


    \begin{micaja}
        3. Demostrar que si se tiene una cadena descendente de subgrupos de la forma
        $$G=G_0 \geq G_1 \geq ...   \geq G_{r-1} \geq G_r,$$ 
        entonces 
        $$|G:G_r| = \prod_{i=0}^{r-1}[Gi:G_{i+1}]$$
    \end{micaja}

    Procederemos a demostrarlo por inducción, el caso base $r=3$ ya está hecho en el apartado anterior. 
    Supongamos ahora cierta la hipótesis de inducción $|G:G_r| = \prod_{i=0}^{r-1}[Gi:Gi+1]$ para $r \geq 3$ y veamos que se cumple para 
    $r+1$. 
    Por la transitividad de ser subgrupo y el teorema de lagrange, llegamos a las siguientes igualdades:
    \begin{align}
        |G| = [G:G_r] |G_r|
    \end{align}
    \begin{align}
        |G| = [G:G_{r+1}] |G_{r+1}|
    \end{align}
    \begin{align}
        |G_r| = [G_r:G_{r+1}] |G_{r+1}|
    \end{align}

Sustituimos en (4) el valor de $|G|$ con (5) y $|G_r|$ con (6), llegando a 

$$[G:G_{r+1}] |G_{r+1}|= [G:G_r] [G_r:G_{r+1}] |G_{r+1}|,
$$
entonces 
$[G:G_{r+1}] = [G:G_r] [G_r:G_{r+1}]$ y utilizando la hipótesis de inducción 
llegamos a 
$$[G:G_{r+1}]= (\prod_{i=0}^{r-1}[Gi:G_{i+1}]) [G_r:G_{r+1}] = \prod_{i=0}^{r}[Gi:G_{i+1}].$$

Probando lo que queríamos. 


\begin{micaja}
    4. Demostrar que si se tiene una cadena descendente de subgrupos de la forma
    $$G=G_0 \geq G_1 \geq ...   \geq G_{r-1} \geq G_r=\{1\},$$ 
    entonces 
    $$|G| = \prod_{i=0}^{r-1}[Gi:G_{i+1}]$$
\end{micaja}

Esto es consecuencia se los apartado (1) y (3) de este ejercio. 

Gracias a (3) tenemos que  $|G:G_r| = \prod_{i=0}^{r-1}[Gi:G_{i+1}]$, 
usando ahora la hipótesis de que $G_r=\{1\}$ y por el apartado (1)

$$|G| = |G:G_r| = \prod_{i=0}^{r-1}[Gi:G_{i+1}]$$

Probando con ello lo que buscábamos. 


\newpage

\section[Ejercicio 21]{Ejercicio 21}  
\begin{micaja}
    Sea G un grupo, $a,b \in G$. 
    \begin{enumerate}
\item  Demuestra que el elemento $b$ y su conjugado $aba^1$ tienen el mismo orden.  
\item Desmostra que o(ba) = o(ab).
    \end{enumerate}
\end{micaja}

$$(aba^{-1})^r = a b^r a^{-1} = 1  \Longleftrightarrow b^r = 1$$. 
Por tanto si r es el orden de alguno de ellos también lo será ara el otro.


\end{document}



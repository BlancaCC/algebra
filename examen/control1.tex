\documentclass[12pt]{article}
\usepackage[utf8]{inputenc}
\usepackage[spanish]{babel}
\usepackage{amsmath} 
\usepackage[dvipsnames]{xcolor}
\usepackage{tikz-cd} % para retículo 


\newenvironment{micaja}
{
    \begin{center}
    \begin{tabular}{|p{0.9\textwidth}|}
    \hline\\
    }   
    {   
    \\\\\hline
    \end{tabular} 
    \end{center}
    }



\title{Control 1}
\author{Blanca Cano Camarero}
\begin{document}
\begin{titlepage}
\maketitle
\tableofcontents
\end{titlepage}

\section{Ejercicio 1}
En el grupo simétrico $S_8$ se consideran los elementos
\begin{equation*}
    \pi = (145)(283)(67) \text{ y } 
    \beta = \begin{pmatrix}
      1 & 2 & 3 & 4 & 5 & 6 & 7 & 8 \\ 
        3 & 5 & 1 & 7 & 8 & 4 & 6 & 2      
    \end{pmatrix}
\end{equation*}
\subsection{Apartado primero}
\begin{micaja}
 Descomponed $\beta$ como producto de ciclos disjuntos y como producto de transposiciones. ¿Cuál es el orden de $\beta$? ¿Cuál es la signatura?
\end{micaja}

 \subsubsection*{Descomposición como ciclos disjuntos}
 Toda permutación se puede expresar de forma única (salvo el orden) de ciclos disjuntos, es este caso:
 
$\beta = (1 3) (2 5 8) (4 7 6)$
\subsubsection*{Descomposición en productos de transposiciones}
Todo ciclo se puede expresar como producto de transposiciones, que no necesariamente tiene porqué ser único (aunque sí siempre de la misma paridad). Para este caso tenemos que:

$\beta = (1 3) (5 8) (2 8) (6 7)(4 6)$
$\beta = (3 4)(1 4) (5 8) (2 8) (6 7) (3 6) (3 4)$

\subsubsection*{Orden de $\beta$}

El orden de una permutación es el mínimo común múltiplo de las longitudes que de los ciclos disjuntos que lo componen, en este caso sería $mcm(2,3,3)=6.$

El orden de $\beta$ es 6.

\subsubsection*{Signatura de $\beta$}

La signatura, signo o paridad de una permutación $s$ es $\sigma(s) = (-1)^n$ donde $n$ es el número de tranposiciones con el que se puede expresar.

Por el teorema anterior sabemos que éste es indiferente de la expresión  que tomemos, ya que la paridad del número de transposiciones de mantiene.

En este caso $n=5$ y por tanto $\sigma(\beta) = (-1).$

\subsection{Apartado segundo}
\begin{micaja}
Hallad un elemento $\alpha  \in S_8$ tal que $\beta = \alpha \pi \alpha^{-1}.$
\end{micaja}
Para esto utilizaremos la siguiente proposición que caracteriza a los conjugados: $\beta (x_0 ... x_r) \beta^{-1} = ( \beta (x_0)... \beta(x_r))$

Por tanto nuestro $\alpha$ buscado cumple la siguiente propiedad:

Para todo $x \in \{1..8\}$ se tiene que $\alpha \pi (x) = \beta(x);$ de lo que deducimos que si $\pi(x)=y$ entonces $\alpha(y) = \beta(x).$

Vamos a darle valores:
\begin{itemize}
\item $\beta(1) = 3$ y $\pi(1)=4$ entontes $\alpha(4)=3$
\item $\beta(2) = 5$ y $\pi(2)=8$ entontes $\alpha(8)=5$
\item $\beta(3) = 1$ y $\pi(3)=2$ entontes $\alpha(2)=1$
\item $\beta(4) = 7$ y $\pi(4)=5$ entontes $\alpha(5)=7$
\item $\beta(5) = 8$ y $\pi(5)=1$ entontes $\alpha(1)=5$
\item $\beta(6) = 4$ y $\pi(6)=7$ entontes $\alpha(7)=4$
\item $\beta(7) = 6$ y $\pi(7)=6$ entontes $\alpha(6)=6$
\item $\beta(8) = 2$ y $\pi(8)=3$ entontes $\alpha(3)=2$
\end{itemize}

\subsection{Apartado tercero}
\begin{micaja}
Si calculamos el producto $\alpha \pi \alpha^{-1}$ para todas las permutaciones de $\alpha \in S_8.$ ¿Cómo podemos caracterizar a las permutaciones obtenidas? ¿Cuántos resultados diferenctes obtenemos?.
\end{micaja}

\subsubsection*{Caracterización}
En virtud de la proposición mencionada en el apartado anterior.
Tenemos lo siguiente:

Sea $s,r$  permutaciones cualquiera, y $s = s_0 s_1 ...  s_m$ es una descomposición en ciclos disjuntos de $s$, entonces tenemos que

$$r s r^{-1} =r s_0 s_1 ...  s_m r^{-1} = (r s_0 r^{-1}) (r s_1 r^{-1}) ... (r s_m r^{-1}.) $$

Por consiguiene cada permutación obenida tendrá tres ciclos disjuntos, dos de ello de longitud 3 y uno de longitud 2. 
Ahora ya podemos aplicar la caracterización del conjugado para cada $r s_i r^{-1}$ con $i \in \{0..m\}.$ y es más si
 $s_i = (x_0^i... x_{w_i}^i)$ entonces sabemos que $(r(x_0^i)... r(x_{w_i}^i)$ deben de formar un ciclo disjunto, del resto, 
ya que de otra forma la aplicación no estaría bien definida. 
\subsubsection*{Cardinalidad.}
Número de permutaciones posibles, las podemos ver gracias a la división en ciclos disjuntos anterior y la biyectividad de las permutaciones: 
será de la forma $(x_0 x_1)( x_2 x_3 x_4)(x_5 x_6 x_7)$ con $x_i \neq x_j$ si $i \neq j$ 

Para el ciclo $(x_0 x_1)$ de dos elementos tenemos $\frac{8!}{(8-2)!}$, lo cual nos deja todavía $8-2$ elementos 
por combinar, para los dos ciclos de tres disjuntos será $\frac{6!}{3!}\frac{3!}{0!}$, pero le tenemos que quitar la mitatad de estos casos
ya que la composición de ciclos disjuntos es conmutativa y tendríamos la misma permutación para 

Por tanto el número total de casos son: $\frac{8!}{6!}\frac{1}{2}(\frac{6!}{3!}\frac{3!}{0!}) = \frac{8!}{2}.$
Ahora bien, estas son todas las posibles, pero ¿podemos obtenerlas todas?
La respuesta a esta pregunta es afirmativa, y la demostración es constructiva, siguiendo 
la misma idea del apartado segundo. 

Veámoslo: 


Sea $\beta = (x_0 x_1)( x_2 x_3 x_4)(x_5 x_6 x_7)$ con $x_i \neq x_j$ si $i \neq j$ y queremos ver que existe $\alpha$
una permutación que cumple que $\beta = \alpha \pi \alpha^{-1}$, 
puesta está definida de manera única para cada elemento.





\subsection{Apartado tercero}

\begin{micaja}
    ¿Es el grupo generado por $\beta$ un subgrupo normal de $S_8$?
\end{micaja}

Por la caracterización de normalidad esto se dará si para cualquier $s \in S_8$
$s <\beta> s^{-1} = <\beta>.$ esto es, que para cualesquiera $b \in <\beta>, s \in S_8$ va a existir 
un $c_{b,s} \in <\beta>$  que cumple que $s bs^{-1} = c_{b,s}.$

Ahora bien $<\beta> = \{id,\beta = (1 3) (2 5 8) (4 7 6), \beta ^2 =(2 8 5)(4 6 2), \beta ^3 =(1 3), \beta ^4 = (2 5 8) (4 7 6), beta^5 = (1 3)(2 8 5)(4 6 2)\}$
y seleccionamos una permutación que no esté en $<\beta>$ y que tenga el mismo número de ciclos disjuntos y de la misma logitud (la caracterización del apartado dos) 
por ejemplo $\gamma = (1 2)(3 5 8)(4 7 6)$ 
y por lo visto en el apartado anterior, sabemos que existirá algún $\alpha \in S_8$ que cumpla que $\gamma = \alpha \beta \alpha^{-1}.$

Y por consiguiente habremos probado que es no es normal. 


 De hecho acabamos de ver más, una caracterizacíon para que sea normal: 
que no se pueda descomponer en ciclos disjuntos, es decir \textbf{que la permutación sea un ciclo}, 
ya que supongamos que existe $\beta$ una permutación que se puede descomponer en ciclos disjuntos, 
$\beta = b_0 b_1...b_n$, tendríamos por commutatividad que 
$\beta ^n = b_0^n b_1^n...b_n^n$. Ahora cogemos una permutación $\gamma$ que sea idéntica a $\beta$ salgo que los dos primeros elementos
de los ciclos disjuntos sea han intercambiado entre sí (esto es si $\beta = (x_0,x_1...)(y_0 y_1..)...(z_0,z_1...)$ entonces $\gamma =(y_0,x_1...)(x_0 y_1..)...(z_0,z_1...)$). 
y por lo visto en el apartado anterior, sabemos que existirá algún $\alpha \in S_8$ que cumpla que $\gamma = \alpha \beta \alpha^{-1}.$

Pero por otra parte hemos visto cómo son los elemento de $<\beta>$, así que $\gamma$ no pertenecerá. 


\newpage

\section{Ejercicio 2}

\subsection{Apartado primero}

\begin{micaja}
    Describi 
\end{micaja}
\end{document}
